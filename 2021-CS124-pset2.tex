
\documentclass[11pt]{article}   
\usepackage{fullpage}
\usepackage{amsfonts}
\usepackage{amssymb}
\usepackage{amsmath}
\usepackage{xcolor}
\usepackage{algorithm}
\usepackage{algorithmic}

\newcommand{\F}{\mathbb{F}}
\newcommand{\np}{\mathop{\rm NP}}
%\newcommand{\binom}[2]{{#1 \choose #2}}
\newcommand{\Z}{{\mathbb Z}}
\newcommand{\vol}{\mathop{\rm Vol}}
\newcommand{\conp}{\mathop{\rm co-NP}}
\newcommand{\atisp}{\mathop{\rm ATISP}}
\renewcommand{\vec}[1]{{\mathbf #1}}
\newcommand{\cupdot}{\mathbin{\mathaccent\cdot\cup}}
\newcommand{\mmod}[1]{\ (\mathrm{mod}\ #1)}  
\def\O{\mathop{\smash{O}}\nolimits}
\def\o{\mathop{\smash{o}}\nolimits}
\newcommand{\e}{{\rm e}}
\newcommand{\R}{{\bf R}}
\renewcommand{\Z}{{\bf Z}}

\setlength{\parskip}{\medskipamount}
\setlength{\parindent}{0in}
%\input{dansmacs}


\begin{document}
	
	\section*{CS 124 Homework 2: Spring 2021}
 		
	\textbf{Your name:} 
		
	\textbf{Collaborators:} 

	\textbf{No. of late days used on previous psets: }\\
	\textbf{No. of late days used after including this pset: }

Homework is due Wednesday 2021-02-17 at 11:59pm ET. You are allowed up to {\bf twelve} (college)/{\bf forty} (extension school) late days through the semester, but the number of late days you take on each assignment must be a nonnegative integer at most {\bf two} (college)/{\bf four} (extension school).

Try to make your answers as clear and concise as possible;
style will count in your grades. Be sure to read and know the collaboration policy in the course
syllabus. Assignments must be submitted in pdf format on Gradescope. If you do assignments by hand, you
will need to scan in your results to turn them in. 

For all homework problems where you are asked to design or give an algorithm, you must prove the correctness
of your algorithm and prove the best upper bound that you can give for the running time. Generally
better running times will get better credit; generally exponential time algorithms (unless specifically asked
for) will receive no or little credit. You should always write a clear informal description of your algorithm
in English. You may also write pseudocode if you feel your informal explanation requires more precision
and detail, but keep in mind pseudocode does NOT substitute for an explanation. Answers that consist
solely of pseudocode will receive little or not credit. Again, try to make your answers clear and concise.

\begin{enumerate}

\item 
{\bf (10 points)}
We saw in lecture that we can find a topological sort of a directed acyclic graph by running DFS
and ordering according to the postorder time
(that is, we add a vertex to the sorted list \emph{after} we visit its out-neighbors).
Suppose we try to build a topological sort by ordering according to the preorder, and 
not the postorder, time.  Give a counterexample to show this doesn't work, and explain why it's a counterexample.  

\item 
{\bf (15 points)}
News from Cambridge, for those of you far away:
every night, snow falls and covers all the sidewalks.
Every morning, the city's lone snow shoveler, Pat, is tasked with clearing all
the sidewalks of snow. Proper snow-shoveling technique requires that 
sidewalks on opposite sides of the same street be shoveled in opposite directions.
(Every street has sidewalks on both sides.)
Give an algorithm to find a snow-shoveling path for Pat that doesn't require
any more walking than necessary---at most once per sidewalk.
(If you have to assume anything about
the layout of the city of Cambridge, make it clear!)
(Your algorithm should work for any city, not just the Cambridge in which Harvard is.)

\item
{\bf (15 points)}
The {\em risk-free currency exchange problem} offers a risk-free way
to make money.  Suppose we have currencies $c_1,\ldots,c_n$.  (For
example, $c_1$ might be dollars, $c_2$ rubles, $c_3$ yen, etc.)  For
various pairs of distinct currencies $c_i$ and $c_j$ (but not
necessarily every pair!) there is an exchange rate $r_{i,j}$ such that
you can exchange one unit of $c_i$ for $r_{i,j}$ units of $c_j$.
(Note that even if there is an exchange rate $r_{i,j}$, so it is possible to
turn currency $i$ into currency $j$ by an exchange, the reverse might not
be true--- that is, there might
not be an exchange rate $r_{j,i}$.)  Now if, because of exchange rate
strangeness, $r_{i,j} \cdot r_{j,i} > 1$, then you can make money
simply by trading units of currency $i$ into units of currency $j$ and
back again.  (At least, if there are no exchange costs.)  This almost
never happens, but occasionally (because the updates for exchange
rates do not happen quickly enough) for very short periods of time
exchange traders can find a sequence of trades that can make risk-free
money.  That is, if there is a sequence of currencies
$c_{i_1},c_{i_2},\ldots,c_{i_k}$ such that $r_{{i_1},{i_2}} \cdot
r_{{i_2},{i_3}} \ldots \cdot r_{{i_{k-1}},{i_k}} \cdot r_{{i_k},{i_1}}
> 1$, then trading one unit of $c_{i_1}$ into $c_{i_2}$ and trading
that into $c_{i_3}$ and so on back to $c_{i_1}$ will yield a profit.

Design an efficient algorithm to detect if a risk-free currency exchange
exists.  (You need not actually find it.)


\item 
{\bf (20 points)}
Give an algorithm to find the lengths of all shortest paths from a given vertex in
a directed graph $G = (V,E)$ where all edge weights are integers between $0$ and $m$, inclusive. 
Your algorithm should work in time $O(|E|+|V|m)$.
(Hint: Modify Dijkstra's algorithm.)

\item 
{\bf (15 points)}
Design an efficient algorithm to find the {\em longest} path in
a directed acyclic graph whose edges have real-number weights. 
 (Partial credit will be given for a solution
where each edge has weight 1; full credit for solutions that handle
general real-valued weights on the edges, including {\em negative} values.)


\item 
{\bf (15 points)}
Suppose that you are given a directed graph $G=(V,E)$ along with
weights on the edges (you can assume that they are all positive).  You
are also given a vertex $s$ and a tree $T$ connecting the graph $G$
that is claimed to be the tree of shortest paths from $s$ that you
would get using Dijkstra's algorithm.  Can you check that $T$ is correct
in linear time?

\item
{\bf (0 points, optional)}\footnote{We won't use this question for grades. Try it if you're interested. 
It may be used for recommendations/TF hiring.}
This exercise is based on the 2SAT problem.  The input to
2SAT is a logical expression of a specific form:  it is the
conjunction (AND) of a set of clauses, where each clause is the
disjunction (OR) of two literals.  (A literal is either a Boolean
variable or the negation of a Boolean variable.)  For example, the
following expression is an instance of 2SAT:
$$(x_1 \, \vee \, \overline{x_2}) \wedge (\overline{x_1} \, \vee \,
\overline{x_3}) \wedge (x_1 \, \vee \, x_2) \wedge (x_4 \, \vee \, \overline{x_3})
\wedge (x_4 \, \vee \, \overline{x_1}).$$
 
A solution to an instance of a 2SAT formula is an assignment of the
variables to the values T (true) and F (false) so that all the clauses
are satisfied-- that is, there is at least one true literal in each
clause.  For example, the assingment $x_1 = T, x_2 = F, x_3 = F, x_4 =
T$ satisfies the 2SAT formula above.
 
Derive an algorithm that either finds a solution to a 2SAT formula, 
or returns that no solution exists. Carefully give a complete
description of the entire algorithm and the running time.

(Hint: Reduce to an appropriate problem.  It may help to consider the
following directed graph, given a formula $I$ in 2SAT: the nodes of
the graph are all the variables appearing in $I$, and their negations.
For each clause $(\alpha \, \vee \, \beta)$ in $I$, we add a directed
edge from $\overline{\alpha}$ to $\beta$ and a second directed edge
from $\overline{\beta}$ to $\alpha$.  How can this be interpreted?)

\item
{\bf (0 points, optional)}\footnote{We won't use this question for grades. Try it if you're interested. 
It may be used for recommendations/TF hiring.}
Give a
complete proof that $\log (n!)$ is $\Theta(n \log n)$.  Hint: you
should look for ways to bound $(n!)$.  Fairly loose bounds will
suffice.
\end{enumerate}
\end{document}
